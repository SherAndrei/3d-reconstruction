\chapter{Теоретический обзор методов восстановления 3D-моделей}

\section{Постановка задачи 3D-восстановления}

В обработке драгоценных камней качество огранки напрямую влияет на стоимость
изделия. Даже небольшие погрешности могут испортить эстетический вид камня.
Так, в ювелирном деле при обработке драгоценных камней мастера располагают
высокоточным и дорогостоящим оборудованием, позволяющим проводить подробное
трехмерное сканирование и создавать точные цифровые модели объектов. Для
минимизации человеческого фактора и для ускорения процесса желателен
нейросетевой метод реконструкции, который мог бы использовать серию снимков для
автоматического получения высокоточной трехмерной модели камня, занимаемый
меньшее время, чем существующие методы. Такая система позволила бы мастеру сразу
сравнивать текущий результат с идеалом и вносить коррективы в процесс обработки.

С другой стороны, в археологии условия работы кардинально отличаются: археологи
часто работают в полевых условиях, где доступ к дорогостоящим сканирующим
устройствам существенно ограничен или полностью отсутствует. При этом задача
создания трехмерных моделей археологических находок, таких как древние черепа,
не требует столь высокой точности, как в ювелирном деле, однако критически важна
для последующего анализа и архивации. В таких условиях возникает потребность в
методах, позволяющих получать качественные трехмерные модели по минимальному
количеству изображений, сделанных при помощи простых и доступных устройств,
таких как обычные камеры смартфонов.

Противопоставляя две эти задачи, следует отметить, что восстановление геометрии
драгоценного камня требует высокой точности и низкой погрешности реконструкции. В то же
время, реконструкция археологического объекта, такого как череп, направлена
прежде всего на целостность и реалистичность итоговой модели, которая будет
использоваться в дальнейших экспертных исследованиях и идентификации, где точное
соответствие исходной форме не всегда является основной целью.

\section{Традиционные и современные подходы}

\subsection{Стереоскопическое изображение (стереопара)}

Стереоскопическая съемка  основана на использовании двух камер, фиксирующих
сцену с разных точек (\cite{ussr1981phototech}).  Такая система позволяет
определить относительное положение предметов по признаку параллакса — смещению
изображения объектов ближнего плана относительно фона при изменении точки
наблюдения.  Основная задача — определить пары соответствующих точек на левом и
правом изображениях и вычислить так называемое смещение (англ.  disparity), т.
е.  разность координат этих точек вдоль горизонтальной оси.

Стереометрические методы опираются на законы эпиполярной геометрии,
согласно которым соответствующие точки лежат на эпиполярных линиях, взаимное
расположение которых описывается фундаментальной матрицей (\cite{Hartley:2003:MVG:861369}).

В простейших реализациях применяются методы блочного сравнения (например, по
критерию суммы квадратов отклонений, коэффициенту взаимной корреляции и др.).
Более совершенные методы используют глобальную оптимизацию (например, на основе
динамического программирования, графовых моделей и пр.). (\cite{kok2019reviewonsterevision}).

\textbf{Преимущества:} Стереофотограмметрия сравнительно легко реализуема и
подробно изучена. Такие методы широко применяются, в том числе в навигации и
системах помощи водителю, а также в киноиндустрии. Современные алгоритмы
достигают высокой точности (точность сопоставления превышает 95\%,
\cite{fsian2022comparisonstereomatchingalgorithms}).

\textbf{Недостатки:} Метод требует наличия перекрывающихся изображений и
наличия текстуры. В однородных или повторяющихся участках возникают ошибки
сопоставления. Также проблемы возникают на границах объектов из-за взаимных
затенений — окклюзий (англ. occlusions), а также при съёмке прозрачных или
зеркальных объектов, где нарушается фотометрическое соответствие.

\subsection{Многовидовая реконструкция (Multi-View Stereo)}

Методы многовидовой реконструкции пространственной формы сцены (англ.
Multi-View Stereo, сокращённо MVS) представляют собой обобщение
стереоскопических методов на случай более чем двух изображений. Целью
является построение плотной трёхмерной модели сцены на основе серии снимков,
полученных с различных точек обзора.

Среди направлений многовидовой реконструкции есть несколько популярных:

\begin{enumerate}
	\item Методы, основанные на построении визуальной оболочки (\cite{10.1109/34.273735}.
	Типичные примеры:
	\begin{enumerate}
		\item методы окрашивания вокселей (\cite{10.5555/794189.794361}),
		\item отсечения несогласованных частей (\cite{10.5555/898435}),
		\item с использованием информации о силуэте объекта (Shape-From-Silhouette, \cite{Matusik2002VHull}).
	\end{enumerate}
	Эти подходы последовательно исключают из объёма те области, проекции которых
	на снимки противоречат наблюдаемому изображению.

	\item Многослойные или гибридные подходы, в которых модель сцены
	представляется совокупностью параллельных пластов (\cite{10.1109/CVPR.1998.698642}).

	\item Поверхностные методы, использующие локальные участки изображения
	(``патчи'') для восстановления облака точек или сеток в пространстве (\cite{10.1109/CVPR.2007.383246}).

\end{enumerate}

Алгоритмы многовидовой реконструкции, как правило, предполагают, что параметры
камер (их расположение и внутренние параметры) известны заранее. Часто для их
оценки используется предварительный этап восстановления структуры сцены из
движения (см. ниже SfM).

К современным решениям, реализующим многовидовую реконструкцию,
относится, например, система COLMAP (\cite{schoenberger2016mvs}), объединяющая этапы определения параметров
камер и построения трёхмерной модели по множеству изображений в едином
автоматизированном процессе.

\textbf{Преимущества:} Многовидовая реконструкция позволяет восстанавливать
детализированные трёхмерные модели сложных объектов, особенно при наличии
большого количества снимков с различными ракурсами. На испытаниях — бенчмарках —
(например, \cite{Knapitsch2017}) классические методы многовидовой реконструкции
демонстрируют высокую точность для непрозрачных объектов. Однако для
восстановления геометрии прозрачных тел, таких как драгоценные камни,
традиционные подходы оказываются малоэффективными без дополнительной информации
или специальных методов.

\textbf{Недостатки:} По сравнению с классическим стерео, многовидовая
реконструкция требует значительно больших вычислительных ресурсов. Поиск
соответствий выполняется сразу между несколькими изображениями, а сама задача
восстановления поверхности — в общем случае нелинейна и решается приближённо.
Также, как и в случае двухкамерной стереосъёмки, фотометрическое согласие
нарушается при съёмке прозрачных или зеркальных поверхностей, что делает такие
объекты трудновосстановимыми без специальных ухищрений.

\subsection{Томографическая реконструкция}

Томографические методы направлены на восстановление структуры объекта
по множеству его проекций. Математически задача сводится к
обращению интегральных преобразований (в частности, преобразования
Радона\footnote{ Иоганн Карл Август Радон (нем. Johann Karl August Radon; 16
декабря 1887, Дечин — 25 мая 1956, Вена) — австрийский математик.  }), решение
которых позволяет приближенно восстановить распределение функции плотности
данного объекта по интегральным прямым вдоль пропущенных через объект лучей
(\cite{book:869357}).

Томографические методы избегают проблемы окклюзии. Они реконструируют трехмерную
геометрию полупрозрачных объектов из ряда теневых изображений, соответствующих
различным положениям источника электромагнитного излучения. Томографию можно
использовать, если среда полупрозрачна относительно длины волны
электромагнитного излучения, используемого для получения данных.  Последнее
требование обычно требует использования рентгеновского излучения, как в
медицинской или инженерной компьютерной томографии (КТ). Рентгеновская
томография основана на дорогостоящем и громоздком оборудовании и не может
использоваться во многих средах из-за соображений безопасности. Тем не менее,
успешные применения в задаче восстановления прозрачных сплошных объектов
(\cite{10.1145/1179849.1179918}) и газов (\cite{IHRKE2006484}).

\textbf{Преимущества:} Такие методы позволяют восстановить не только внешнюю
поверхность, но и внутренние оптические свойства объекта — в частности,
показатель преломления в разных частях. Это делает возможным получение
высокоточной информации даже для оптически сложных объектов, где классические методы
неэффективны.

\textbf{Недостатки:} Требуются специальные условия и оборудование,
известный фон или шаблон позади объекта. Классические алгоритмы томографии
предполагают большое число проекций (десятки и сотни) и чувствительны
к шуму в измерениях.

\subsection{Восстановление структуры из движения (Structure from Motion)}

Методы восстановления структуры из движения (англ. Structure from Motion,
 SfM) предназначены для одновременного определения пространственного
расположения объектов сцены и положения съёмочной камеры по множеству
изображений, как правило, полученных с помощью одной перемещающейся камеры.

В отличие от стереоскопических или многовидовых методов, которые обычно
предполагают известные параметры камер и работают с несколькими фиксированными
изображениями, SfM начинает с неупорядоченного
набора изображений и в процессе определяет как положение камеры на каждом кадре,
так и пространственные координаты ключевых точек сцены. При этом изначально
формируется разреженное облако точек (англ. point cloud) — то есть восстанавливаются не все детали
сцены, а только те, которые можно устойчиво наблюдать на нескольких
изображениях (\cite{10.1109/CVPR.2016.4454}).

В результате работы метода восстанавливаются положения всех камер и
пространственное облако точек, отражающее структуру наблюдаемой сцены. Такой
подход широко применяется в фотограмметрии, особенно при обработке любительских
или архивных фотоснимков. Современные программные обеспечения (напр., Bundler
(\cite{10.1145/3596711.3596766}) VisualSfM (\cite{10.1109/3DV.2013.25}), COLMAP)
способны автоматически восстанавливать трёхмерную структуру по произвольному
набору изображений, полученных даже не калиброванными камерами.

\textbf{Преимущества:} методы восстановления структуры из движения позволяют
строить трёхмерные модели по неупорядоченным наборам изображений, не требуя
предварительной калибровки. Они хорошо масштабируются и позволяют обрабатывать
большие массивы фотоданных — вплоть миллиона кадров (\cite{10.1109/CVPR.2015.7298949}).

\textbf{Недостатки:} итогом является разреженное представление сцены — т. е.
лишь ограниченное количество пространственных точек. Для получения плотной модели
требуется дальнейшее уплотнение, например, с использованием многовидовой
реконструкции. Кроме того, методы SfM чувствительны к качеству входных
изображений. Если объект однороден, лишён текстурных деталей или является
прозрачным, найти устойчивые соответствия между изображениями становится
затруднительно или невозможно.

\subsection{Фотометрические методы восстановления формы}

Фотометрические методы основаны на анализе изменения яркости в различных точках
изображения объекта при различном освещении, при этом положение камеры остаётся
фиксированным. В отличие от методов стереоскопии и многовидовой реконструкции,
фотометрические подходы используют светотеневую информацию для восстановления
нормалей поверхности.

Среди наиболее известных подходов:

\begin{itemize}
	\item Восстановление по тени (англ. shape from shading, SFS) —
	метод, использующий изменение яркости на изображении, вызванное градиентами
	освещённости. Помимо чувствительности к шумам предполагается знание положения источника света и
	отражательных свойств поверхности (\cite{Horn1989SFS}).
	\item Фотометрическое стерео (англ. photometric stereo method, PSM) — метод, при котором
	объект освещается с разных направлений, но камера остаётся неподвижной. По
	изменению яркости одной и той же точки объекта под разным освещением
	вычисляется нормаль к поверхности в этой точке. Такой подход позволяет детально
	восстановить локальную геометрию — микрорельеф поверхности
	(\cite{10.1117/12.7972479}).
\end{itemize}

Оба метода предполагают, что поверхность объекта — ламбертова \footnote{Иоганн
Генрих Ламберт (нем. Johann Heinrich Lambert; 26 августа 1728, Мюлуз, Эльзас —
25 сентября 1777, Берлин) — немецкий физик, врач, философ, математик и
астроном.}, т. е.  отражает свет равномерно во всех направлениях, как матовая.
Отражение от таких поверхностей называют диффузным.

Прямое применение этих методов к прозрачным объектам затруднено. Яркость точек
на изображении в этом случае определяется не отражением, а преломлением света,
прошедшего через объект. Это нарушает основное предположение фотометрических
методов о связи между направлением нормали и яркостью. Тем не менее существуют
расширения алгоритмов, которые позволяют работать и с более сложными поверхностями.
Например, правильно подобранные камеры и их положения помогают восстанавливать поверхности,
которые не удовлетворяют закону Ламберта (\cite{McGunnigle-2012}).

\textbf{Преимущества:} Методы позволяют точно восстановить микрорельеф объектов,
особенно при диффузном отражении. Не требуются дорогостоящее оборудование,
делая пригодным использование в домашних условиях.

\textbf{Недостатки:} Предложенные алгоритмы не работают напрямую с прозрачными,
зеркальными или частично преломляющими материалами. Требуют строгого контроля
освещения и знаний об отражательных свойствах материала. PSM восстанавливает
лишь видимую камерой часть объекта, речи о полном восстановлении 3D модели не идет.

\subsection{Активные методы трёхмерного сканирования}

Активные методы восстановления трёхмерной формы объекта используют управляемые
источники излучения — свет, лазер или радиоволны — для освещения сцены и анализа
её отклика. В отличие от пассивных методов (таких как стереозрение,
фотометрические и томографические подходы), активные системы формируют
изображение или глубинную карту за счёт взаимодействия между излучаемым и
отражённым световым сигналом.

Такие методы особенно эффективны в условиях контролируемой съёмки, при слабом
или нестабильном внешнем освещении, а также при необходимости высокой точности
измерений.

Известны несколько классов активных методов:

\begin{itemize}
	\item Сканирование лазерным лучом — система испускает узконаправленный луч
	(чаще всего в инфракрасном диапазоне), который отражается от поверхности
	объекта и регистрируется сенсором. По времени возврата луча (время
	пролёта, Time-of-Flight, ToF) или по углу отражения можно определить
	расстояние до точки. Такой подход позволяет построить высокоточные карты
	глубины (\cite{10.1109/CVPR.2010.5540082}).
	\item Структурированное освещение — на поверхность объекта проецируется
	заранее известный шаблон (сетка, полосы, фазовый узор и т.п.). Камера
	фиксирует искажения шаблона, вызванные формой объекта. На основе анализа
	деформаций этого узора восстанавливается трёхмерная геометрия сцены. Интересным
	примером является использование псеводслучайных кодов. Суть подхода
	заключается в проецировании на объект двумерной матрицы точек, закодированных
	по специальному закону, обеспечивающему уникальную локализацию каждой точки на
	изображении на основе её ближайшего окружения. Авторы показали, что даже при наличии
	окклюзии, эффективность сопоставления точек остаётся высокой (\cite{10.1109/34.667888}).
	\item Радиолокационное синтезирование апертуры (англ. Synthetic Aperture
	Radar, SAR) применяется в дистанционном зондировании и геофизике. Метод
	позволяет формировать карты рельефа с высокой точностью, даже сквозь
	облачность или в условиях недостаточной освещённости (\cite{Antipov1988}).
\end{itemize}

Прямое применение активных методов к прозрачным телам (например, стекло,
хрусталь, драгоценные камни) затруднено. Отражение и преломление светового
сигнала на таких поверхностях носит сложный характер: большая часть излучения
проходит сквозь объект или многократно отражается внутри него. Это нарушает
ключевое предположение большинства методов — однозначное соответствие между
точкой на поверхности и зафиксированным откликом.

Тем не менее, существуют эмпирические решения, например, поверхность прозрачного
объекта покрывается матирующим (оптически рассеивающим) спреем, что временно делает
её диффузной.

\textbf{Преимущества:} Применение таких алгоритмов гарантируют высокую точность
восстановления формы в условиях нестабильного освещения.

\textbf{Недостатки:} Для обеспечения хороших результатов требуется дорогостоящее
и громоздкое оборудование, тем самым делая затрудненным применение в полевых
условиях и при свободной съёмке (например, в музейной фотографии).

\section{Нейросетевые методы 3D-реконструкции}

За последние несколько лет появились методы, использующие машинное обучение для
реконструкции 3D-сцены только по фотографиям и известным параметрам камеры (без
дополнительных сенсоров). Здесь мы рассмотрим две большие категории: (1)
нейросетевые аналоги классических методов (например, обучаемый MVS), и (2) новые
подходы с неявными репрезентациями сцены, такие как Neural Radiance
Fields.

\subsection{Обучаемые многовидовые методы восстановления (Learning-based MVS)}

Современные достижения в области машинного обучения позволили расширить
традиционные методы многовидовой реконструкции с использованием машинного
обучения и нейросетей. Такие методы получили название обучаемых многовидовых
методов (от англ. Learning-based Multi-View Stereo).

Первым поколением нейросетевых многовидовых методов были объёмные подходы, такие
как SurfaceNet (\cite{ji2017surfacenet}) и LSM
(\cite{kar2017learningmultiviewstereomachine}). Эти методы формируют трёхмерный
объём сцены (воксельную сетку), в котором каждый элемент описывает
согласованность признаков, извлечённых из нескольких изображений. Для анализа
такого объёма применяются трёхмерные свёрточные сети (англ. convolutional neural
network, CNN). Однако объёмное представление крайне неэффективно с точки зрения
памяти и потому подходит только для реконструкции небольших сцен.

Значительным шагом вперёд стал метод MVSNet
(\cite{yao2018mvsnetdepthinferenceunstructured}), в котором вместо вокселей
используется глубинное (depth-based) представление. Метод получает на вход одно
опорное изображение и несколько дополнительных, извлекает из них признаки, и
проектирует их в единую систему координат с помощью дифференцируемого
преобразования — гомографии. На основе этих проекций строится трёхмерный
объём согласия (англ. cost volume) — по сути, для каждой гипотетической глубины
оценивается, насколько хорошо согласуются признаки с разных ракурсов. Этот объём
обрабатывается 3D-свёртками, после чего получается карта глубины, которая при
необходимости уточняется дополнительным модулем (refinement).

Из-за высокой вычислительной сложности (особенно по памяти), в дальнейшем
появились модификации MVSNet, направленные на повышение эффективности:

\begin{itemize}
	\item Многошаговые (multi-staged) методы, например, CasMVSNet
	(\cite{gu2020cascadecostvolumehighresolution}), используют стратегию ``от
	грубого к точному'' (coarse-to-fine): сначала восстанавливается грубая глубина
	с большим шагом, а затем она уточняется на следующих уровнях с меньшим шагом.
	Это позволяет уменьшить требования к памяти, но в некоторых случаях начальное
	грубое приближение может содержать ошибки, которые трудно скорректировать на
	следующих этапах.
	\item Рекуррентные методы, например, R-MVSNet
	(\cite{yao2019recurrentmvsnethighresolutionmultiview}), обходят необходимость
	хранения всего объёма согласия путём поочерёдной обработки глубинных сечений с
	использованием рекуррентных нейросетей (англ. Recurrent Neural Network, RNN).
	Это позволяет существенно расширить диапазон обрабатываемых глубин и снизить
	потребление памяти, сохранив высокое качество.
\end{itemize}

\textbf{Преимущества:}
Обучаемые методы способны автоматически извлекать информативные признаки и
выполнять реконструкцию без необходимости ручной настройки параметров. Они
демонстрируют высокую точность, устойчивость к шуму и частичным окклюзиям, так
как могут использовать статистические закономерности форм и текстур, усвоенные
из обучающих данных, значительно опережая традиционные подходы
(\cite{10.1109/CVPR.2017.272}). Такие модели, как правило, устойчивы к
частичному отсутствию информации и могут работать с переменным числом входных
изображений. Кроме того, они хорошо масштабируются и позволяют адаптировать
систему под конкретные типы сцен, например, интерьеры помещений или уличные
пейзажи.

\textbf{Недостатки:}
Главным недостатком является высокая вычислительная сложность: хранение объёма
согласия и 3D-свёртки требуют значительных объёмов видеопамяти (несколько
гигабайт даже для одного кадра), а также производительного графического
ускорителя (GPU). Кроме того, такие методы чувствительны к отклонениям от
обучающего распределения: устойчивость к шуму зависит от наличия подобных
искажений в обучающей выборке: если они не встречались, качество существенно
падает.

\subsection{Генеративные нейросетевые модели для прямого предсказания}

Ещё одно направление – обучать нейросети, которые прямо по одному или нескольким
изображениям выдают 3D-модель объекта, основываясь на знании, полученном из
большого корпуса данных. Акцент делается на непосредственном предсказании (англ.
inference). Такие модели называют генеративными.

Одним из первых значимых шагов в развитии генеративных нейросетевых подходов по
восстановлению трёхмерных моделей по изображениям стала модель 3D-R2N2
(\cite{choy20163dr2n2unifiedapproachsingle}). Используется рекуррентная
нейронная сеть, которая способна агрегировать признаки с нескольких изображений
и на их основе построить трёхмерную воксельную модель объекта. Основной
недостаток подхода — ограниченное разрешение выходной модели, обычно не
превышающее объёмов \(32^3\) или \(64^3\) вокселей, теряя тем самым детали
изучаемого объекта.

Чтобы преодолеть это ограничение, была предложена модель AtlasNet
(\cite{groueix2018atlasnetpapiermacheapproachlearning}). В ней объект
представляется в виде набора полосок. Авторы сравнивают генерацию объекта c
способом, похожим на наложение полосок бумаги на форму, чтобы сформировать
папье-маше. Такой подход позволил повысить детализацию, но ввёл новую
трудность: при несовпадении границ ``волокон'' возникают артефакты, а сама процедура
параметризации усложняет обучение.

В том же году была представлена модель Pixel2Mesh
(\cite{wang2018pixel2meshgenerating3dmesh}), которая вместо полосного
представления использовала деформацию единой сетки (например, сферы), уточняемую
с помощью графовых сверточных сетей (Graph-CNN — обобщение обычных свёрток на
графы). Это обеспечило гладкость и связность результирующей поверхности, однако
модель хуже справляется с топологически сложными или тонкими деталями.

Следующим этапом эволюции генеративных моделей является модель Pix2Vox
(\cite{Xie_2019}), сочетающая в себе идеи агрегирования многовидовых признаков и
последующего преобразования их в трёхмерное воксельное представление высокого
качества. Это позволило достичь более полной и точной реконструкции, но модель
унаследовала высокое потребление памяти и ограничение по разрешению.

Появились на свет диффузионные модели (англ. Diffusion Probabilistic Models,
DPMs), такие как 3DiM (\cite{watson2022novelviewsynthesisdiffusion}), которые
обучаются восстанавливать 3D-представления путём пошаговой генерации (по
аналогии с процессами тепловой диффузии в физике). Эти модели показывают высокий
уровень фотореализма, но требуют большого количества обучающих данных и
значительных ресурсов. Кроме того, они пока недостаточно исследованы для
использования в точных инженерных задачах.

\textbf{Преимущества:} генеративные модели обеспечивают высокую скорость
предсказания (мгновенный результат после обучения), хорошо используют контекст и
обученные знания о формах. Например, сеть может восстановить невидимую часть
объекта, опираясь на его типичную структуру, если такие формы были в обучающем
наборе.

\textbf{Недостатки:} обобщающая способность таких моделей ограничена: они хорошо
работают только в пределах тех категорий, на которых обучались (например,
мебель, автомобили, самолёты и т.\,п.). Кроме того, качество реконструкции
быстро падает за пределами обучающего распределения.  Также выходное разрешение
часто остаётся ограниченным: воксельные методы грубы, а сеточные — ограничены
числом вершин.

\subsection{Neural Radiance Fields (NeRF) и его расширения}

Одним из наиболее значимых достижений в области трёхмерного представления сцены
стала модель Neural Radiance Fields (NeRF). В отличие от классических подходов,
явно восстанавливающих геометрию или карту глубины, NeRF использует
неявное представление — непрерывную функцию, приближаемую нейросетью, которая
задаёт поведение сцены в каждой точке пространства.

Нейронная сеть представляет из себя многослойный полносвязный перцептрон (англ.
Multilayer Perceptron, MLP), в отличие от своих свёрточных предшественников.
Обучение происходит на конкретной сцене без посторонних данных. Таким образом,
NeRF — это не столько ``обучаемая модель'' в традиционном понимании, сколько
процедура численного интерполирования для одной сцены.

Так как NeRF обучается на конкретной сцене, случайный шум может быть ``усреднён''
моделью, которая скорее аппроксимирует гладкие функции. Однако систематические
артефакты (например, блики) могут быть интерпретированы как структура
непосредственного объекта, искажающая распределение плотности. Для борьбы с
переобучением вводят регуляризацию или используют раннюю остановку оптимизации.

\textbf{Преимущества:}
Модель NeRF обеспечивает фотореалистичный синтез изображений с новых ракурсов
при наличии ограниченного числа входных снимков. Сеть эффективно интерполирует
структуру между видами, включая мелкие и сложноустроенные элементы сцены, такие
как тонкие ветви или прозрачные оболочки.

\textbf{Недостатки:}
\begin{itemize}
	\item NeRF работает только со статичными сценами и требует точных данных о
	положении камер. Если позиции камер были определены неточно, результат
	получался размытым — из-за этого позже появились методы вроде BARF
	(\cite{lin2021barfbundleadjustingneuralradiance}), которые подбирают и саму
	сцену, и положение камер вместе.
	\item NeRF плохо справляется с преломлениями и отражениями: такие
	эффекты модель воспринимает как «свечение» внутри объекта, что приводит к
	артефактам. Прозрачные объекты создают те же проблемы, что и в классических
	методах Multi-View Stereo (MVS): границы получаются неточными, а внутренняя
	структура искажается. Позднее было предложено множество модификаций, успешно
	обходящих эту проблему. Например, NeRFReN
	(\cite{guo2022nerfrenneuralradiancefields}) делит сцену на два поля:
	прозрачную компоненту, моделирующую прямое прохождение света, и отражённую —
	зависящую от направления взгляда.
	\item Исходный NeRF чрезвычайно медленный: обучение на одной сцене может занимать
	несколько десятков часов, иногда дни. Позднее появились ускоренные версии, не уступающие
	по качеству:
	\begin{itemize}
		\item Решение Instant-NGP (\cite{M_ller_2022}) использует маленькую MLP, храня признаки в хеш-таблицах.
		С помощью этого алгоритма качественная сцена может быть обучена за 5-30 секунд.
		Важным минусом является низкоуровневое использование проприетарных технологий (NVIDIA CUDA \footnote{\url{https://developer.nvidia.com/cuda-toolkit}}),
		что делает невозможным использование метода при отсутствии соответствующих графических
		ускорителей.
		\item Алгоритм Plenoxels (\cite{yu2021plenoxelsradiancefieldsneural}) не использует
		нейросети — только явное представление сцены в виде воксельной сетки с
		интерполяцией и регуляризацией. Алгоритм достигает сопоставимого качества
		визуализации, но обучается в 100 раз быстрее (всего 11 минут против 2 дней у
		NeRF на тех же сценах). Из минусов меньшая гибкость при обобщении, т.к.
		отсутствует обучаемая структура, и большое потребление памяти для
		хранения градиентов.
		\item TensoRF (\cite{chen2022tensorftensorialradiancefields}) использует
		четырехмерный тензор признаков и применяет введенную авторами
		векторно-матричную тензорную декомпозицию. Алгоритм дает более высокое качество отрисовки,
		требуя меньше времени (в 70 раз быстрее NeRF) на обучение и занимая меньше памяти.
	\end{itemize}
\end{itemize}

\subsection{Неявные представления поверхности}

Наряду с методами объёмного моделирования, такими как NeRF, получили развитие
подходы, в которых поверхность объекта восстанавливается напрямую, в виде
неявной функции. Основным преимуществом неявных нейронных представлений является
их гибкость в представлении поверхностей с произвольными формами,
а современные подходы не требуют сеточной структуры (воксельной или треугольной).

Одним из первых прорывных методов неявного восстановления поверхности стал
Differentiable Volumetric Rendering (DVR)
(\cite{niemeyer2020differentiablevolumetricrenderinglearning}). В этом подходе
геометрия сцены представляется воксельной сеткой или функцией расстояния (англ.
Signed distance function, SDF), а визуализация осуществляется за счет диффузного
отражения. DVR использует простую дифференцируемую схему обучения и не требует
предобучения.  Однако он ограничен в точности геометрии и не способен корректно
моделировать сложные оптические объекты, такие как зеркальные отражения или блеск.

Развитием этой идеи стал метод IDR (Implicit Differentiable Renderer)
(\cite{yariv2020multiviewneuralsurfacereconstruction}), в котором сцена задаётся
через неявную функцию $g(\mathbf{x})$, реализованную с помощью полносвязной
нейросети (MLP). Поверхность объекта определяется линией уровня $g(\mathbf{x}) =
0$, а цвет пикселя рассчитывается как функция положения точки, её нормали и
направления взгляда, с учётом отражательных свойств материала. На вход модель
получает изображения сцены с маской (объект без заднего фона) и приближённые
положения камер, которые могут быть получены с помощью другого программного
обеспечения, например, COLMAP. Все компоненты модели — поверхность, материалы,
рендеринг и даже параметры поз камер — подбираются одновременно на основе
функции ошибки между синтезированными и настоящими изображениями. Основным
ограничением IDR остаётся вычислительная сложность: оптимизация проводится
отдельно для каждой сцены, а обучение чувствительно к качеству масок и начальным
условиям.

В попытке объединить преимущества поверхностного описания и объёмного
рендеринга, был предложен метод UNISURF
(\cite{oechsle2021unisurfunifyingneuralimplicit}). Он также использует неявное
представление поверхности через функцию расстояния, но визуализацию производит
по аналогии с объёмными методами — через вероятностную интерпретацию занятости
пространства. Это повышает устойчивость обучения и упрощает рендеринг. Однако
UNISURF уступает IDR в точности определении границ поверхности и может смазывать
детали.

Дальнейшим шагом стал метод Neural Implicit Surfaces (NeuS)
(\cite{wang2023neuslearningneuralimplicit}). Он аналогично основан на SDF, но вводит
специальную конструкцию функции плотности $\sigma$, которая позволяет применять
классический объёмный рендеринг (аналогично NeRF), при этом обеспечивая резкое и
корректное моделирование поверхности. Таким образом, NeuS объединяет гладкость и
устойчивость NeRF с точностью IDR. Главным недостатком метода остаётся
чувствительность к конфигурации плотности: при некорректном выборе функции
$\sigma$ возможны потери точности или нестабильность обучения.

Большинство перечисленных методов ориентированы на восстановление отражённого
света и поэтому плохо применимы к прозрачным телам. Для решения этой проблемы
был представлен алгоритм NeTO (Neural Transparent Object)
(\cite{li2023netoneuralreconstructiontransparentobjects}), адаптированный к
реконструкции прозрачных объектов. NeTO использует SDF-представление геометрии,
а визуализацию реализует через трассировку
лучей при преломлении (англ. refraction-tracing), опираясь на закон Снеллиуса\footnote{Виллеброрд Снелл ван Ройен (лат.
Willebrordus Snellius) — нидерландский математик, сформулировавший закон
преломления света.} о преломлении. При оптимизации отслеживаются как входящие, так и
выходящие из объекта лучи, включая случаи самозатенения (self-occlusion). Это
позволяет точно реконструировать форму даже из небольшого количества изображений
(меньше 20 снимков), в отличие от более раннего метода DRT
(\cite{Lyu_2020}), который использовал явную сеточную модель поверхности, но не
учитывал эффектов самозатенения и требовал десятки видов для сопоставимого
качества.

\textbf{Преимущества:} Методы с неявным представлением поверхности позволяют
восстанавливать геометрию сцены напрямую и с высокой точностью. В отличие от
моделей, работающих с объемом, где плотность сцены может быть размазанной, здесь
оптимизация стремится к резкой, чётко определённой границе объекта.
Дополнительно, такие методы могут одновременно оценивать свойства материалов —
например, отражательную способность — что особенно полезно при моделировании
объекта. В расширенных версиях (таких как NeTO) становится возможным учитывать
даже преломление, что открывает путь к моделированию стекла, жидкостей и
драгоценных камней.

\textbf{Недостатки:} Основной недостаток — высокая вычислительная нагрузка.
Обучение требует поиска минимума в большом пространстве параметров (геометрия,
освещение, камеры), что может приводить к локальным минимумам или переобучению.
При плохом качестве входных изображений или ошибках в калибровке
может быть трудно различить, где ошибка вызвана геометрией, а где — световыми
эффектами. Для прозрачных объектов требуется специализированная модель
рендеринга, иначе результат будет физически недостоверным. Несмотря на это,
такие методы всё чаще показывают конкурентные или превосходящие результаты по
сравнению с классическими многовидовыми подходами.

