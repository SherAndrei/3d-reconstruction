\chapter{Теоретический обзор методов восстановления 3D-моделей}

\section{Постановка задачи 3D-восстановления}

Качественно выполненная огранка драгоценного камня существенно повышает
стоимость ювелирного изделия, в то же время неправильная огранка негативно
влияет на эстетическое восприятие. Для минимизации человеческого
фактора и для ускорения процесса желателен нейросетевой метод реконструкции,
который мог бы использовать серию снимков для автоматического получения
высокоточной трехмерной модели камня. Это позволяет мастеру мгновенно сравнивать
получаемый результат с исходной моделью и оперативно корректировать действия для
достижения идеальной формы.

В то же время, в археологии имеются принципиально иные предпосылки для задачи
восстановления трехмерных моделей. Здесь трехмерная реконструкция нужна не
столько для достижения точности соответствия эталонной геометрии, сколько для
идентификации личности по черепу, изучения анатомических особенностей,
проведения антропологических исследований и судебно-медицинских экспертиз. Целью
является реконструкция полной и детализированной модели объекта, даже если
исходные данные неполны или повреждены. Метод должен быть устойчив к шумам и
потерям информации, и при этом позволять визуализировать максимально
информативную трехмерную модель.

Противопоставляя две эти задачи, следует отметить, что восстановление геометрии
драгоценного камня требует высокой точности и низкой погрешности реконструкции. В то же
время, реконструкция археологического объекта, такого как череп, направлена
прежде всего на целостность и реалистичность итоговой модели, которая будет
использоваться в дальнейших экспертных исследованиях и идентификации, где точное
соответствие исходной форме не всегда возможно и не является основной целью.

Таким образом, для эффективного решения обеих задач необходимы различные подходы
к трехмерному моделированию: для огранки камней приоритетом является
высокоточная геометрическая реконструкция, в то время как в археологической
задаче реконструкции черепов важна полная, визуально реалистичная модель,
способная корректно отразить исходную структуру и особенности, даже в условиях
частичной потери исходных данных.

\section{Традиционные и современные подходы}

\subsection{Стереоскопическое изображение (стереопара)}

Стереоскопическая съемка  основана на использовании двух съёмочных систем (или
двух изображений), фиксирующих сцену с разных точек (\cite{ussr1981phototech}).
Такая система позволяет определить относительное положение предметов по признаку
параллакса — смещению изображения объектов ближнего плана относительно фона при
изменении точки наблюдения.  Основная задача — определить пары соответствующих
точек на левом и правом изображениях и вычислить так называемое смещение (англ.
disparity), т. е.  разность координат этих точек вдоль горизонтальной оси.
Расстояние до объекта (глубина сцены) $Z$ обратно пропорционально смещению: $Z =
\frac{f \cdot B}{d}$, где $f$ — фокусное расстояние объектива, $B$ — база
(расстояние между оптическими осями камер), $d$ — смещение.

Стереометрические методы опираются на законы эпиполярной геометрии,
согласно которым соответствующие точки лежат на эпиполярных линиях, взаимное
расположение которых описывается фундаментальной матрицей (\cite{Hartley:2003:MVG:861369}).

В простейших реализациях применяются методы блочного сравнения (например, по
критерию суммы квадратов отклонений, коэффициенту взаимной корреляции и др.).
Более совершенные методы используют глобальную оптимизацию (например, на основе
динамического программирования, графовых моделей и пр.). Каждый из предложенных
развитий метода несет свои достоинства и недостатки
(\cite{kok2019reviewonsterevision}). Рассмотрим достоинства и ограничения
для алгоритма в общем случае.

\textbf{Преимущества:} стереофотограмметрия сравнительно легко реализуема и
хорошо изучена. Такие методы широко применяются, в том числе в навигации и
системах помощи водителю, а также в киноиндустрии. Современные алгоритмы
достигают высокой точности (точность сопоставления превышает 95\%,
\cite{fsian2022comparisonstereomatchingalgorithms}).

\textbf{Недостатки:} метод требует наличия перекрывающихся изображений и
наличия текстуры. В однородных или повторяющихся участках возникают ошибки
сопоставления. Также проблемы возникают на границах объектов (из-за взаимных
затенений, англ. occlusions), а также при съёмке прозрачных или зеркальных
объектов, где нарушается фотометрическое соответствие. Методы
стереофотограмметрии, как правило, предполагают ламбертов характер отражения,
что делает их неприменимыми для оптически сложных поверхностей.

\subsection{Многовидовая реконструкция (Multi-View Stereo)}

Методы многовидового восстановления пространственной формы сцены (англ.
Multi-View Stereo, сокращённо MVS) представляют собой обобщение
стереоскопических методов на случай более чем двух изображений ($>2$). Целью
является построение плотной трёхмерной модели сцены на основе серии снимков,
полученных с различных точек обзора.

Среди направлений многовидовой реконструкции есть несколько популярных:

\begin{enumerate}
	\item Методы, основанные на построении визуальной оболочки (\cite{10.1109/34.273735}.
	Типичные примеры:
	\begin{enumerate}
		\item методы окрашивания вокселей (\cite{10.5555/794189.794361}),
		\item отсечения несогласованных частей (\cite{10.5555/898435}),
		\item с использованием информации о силуэте объекта (Shape-From-Silhouette, \cite{Matusik2002VHull}).
	\end{enumerate}
	Эти подходы последовательно исключают из объёма те области, проекции которых
	на снимки противоречат наблюдаемому изображению.

	\item Многослойные или гибридные подходы, в которых модель сцены
	представляется совокупностью параллельных пластов (\cite{10.1109/CVPR.1998.698642}).

	\item Поверхностные методы, использующие локальные участки изображения
	(`патчи') для восстановления точек /
	сеток в пространстве (\cite{10.1109/CVPR.2007.383246}).

\end{enumerate}

\textbf{Преимущества:} многовидовая реконструкция позволяет восстанавливать
детализированные трёхмерные модели сложных объектов, особенно при наличии
большого количества снимков с различными ракурсами. При благоприятных условиях
(матовые объекты, выраженная текстура, высокая точность съёмки) точность может
соперничать с результатами лазерного сканирования.

\textbf{Недостатки:} по сравнению с классическим стерео, многовидовая
реконструкция требует значительно больших вычислительных ресурсов. Поиск
соответствий выполняется сразу между несколькими изображениями, а сама задача
восстановления поверхности — в общем случае нелинейна и решается приближённо.
Также, как и в случае двухкамерной стереосъёмки, фотометрическое согласие
нарушается при съёмке прозрачных или зеркальных поверхностей, что делает такие
объекты трудновосстановимыми без специальных ухищрений.

Алгоритмы многовидовой реконструкции, как правило, предполагают, что параметры
камер (их расположение и внутренние параметры) известны заранее. Часто для их
оценки используется предварительный этап восстановления структуры сцены из
движения (см. ниже SfM).

К современным решениям, реализующим многовидовую реконструкцию,
относится, например, система COLMAP (\cite{schoenberger2016mvs}), объединяющая этапы определения параметров
камер и построения трёхмерной модели по множеству изображений в едином
автоматизированном процессе.

На испытаниях — бенчмарках — (например, \cite{Knapitsch2017}) классические методы многовидовой
реконструкции демонстрируют высокую точность для непрозрачных объектов. Однако
для восстановления геометрии прозрачных тел, таких как драгоценные камни,
традиционные подходы оказываются малоэффективными без дополнительной информации
или специальных методов.

\subsection{Томографическая реконструкция}

Томографические методы направлены на восстановление структуры объекта
по множеству его проекций. Математически задача сводится к
обращению интегральных преобразований (в частности, преобразования
Радона\footnote{ Иоганн Карл Август Радон (нем. Johann Karl August Radon; 16
декабря 1887, Дечин — 25 мая 1956, Вена) — австрийский математик.  }), решение
которых позволяет приближенно восстановить распределение функции плотности $f(x,y,z)$ объекта
по интегральным прямым вдоль пропущенных через объект лучей (\cite{book:869357}).

Томографические методы избегают проблемы окклюзии. Они реконструируют трехмерную
геометрию полупрозрачных объектов из ряда теневых изображений, соответствующих
различным положениям источника электромагнитного излучения. Томографию можно
использовать, если среда полупрозрачна относительно длины волны электромагнитного
излучения, используемого для получения данных.  Последнее требование обычно
требует использования рентгеновского излучения, как в медицинской или инженерной
компьютерной томографии (КТ). К сожалению, рентгеновская томография основана на
дорогостоящем и громоздком оборудовании и не может использоваться во многих
средах из-за соображений безопасности. Тем не менее, успешные применения
в задаче восстановления прозрачных сплошных объектов (\cite{10.1145/1179849.1179918}) и
газов (\cite{IHRKE2006484}).

\textbf{Преимущества:} такие методы позволяют восстановить не только внешнюю
поверхность, но и внутренние оптические свойства объекта — в частности,
показатель преломления в разных частях. Это делает возможным получение
высокоточной информации даже для оптически сложных объектов, где классические методы
неэффективны.

\textbf{Недостатки:} требуются специальные условия и оборудование,
известный фон или шаблон позади объекта. Классические алгоритмы томографии
предполагают большое число проекций (десятки и сотни) и чувствительны
к шуму в измерениях.

\subsection{Восстановление структуры из движения (Structure from Motion)}

Методы восстановления структуры из движения (англ. Structure-from-Motion,
 SfM) предназначены для одновременного определения пространственного
расположения объектов сцены и положения съёмочной камеры по множеству
изображений, как правило, полученных с помощью одной перемещающейся камеры.

В отличие от стереоскопических или многовидовых методов, которые обычно
предполагают известные параметры камер и работают с несколькими фиксированными
изображениями, SfM начинает с неупорядоченного
набора изображений и в процессе определяет как положение камеры на каждом кадре,
так и пространственные координаты ключевых точек сцены. При этом изначально
формируется разреженное облако точек — то есть восстанавливаются не все детали
сцены, а только те, которые можно устойчиво наблюдать на нескольких
изображениях (\cite{10.1109/CVPR.2016.4454}).

В результате работы метода восстанавливаются положения всех камер и
пространственное облако точек, отражающее структуру наблюдаемой сцены. Такой
подход широко применяется в фотограмметрии, особенно при обработке любительских
или архивных фотоснимков. Современные программные обеспечения (напр., Bundler
(\cite{10.1145/3596711.3596766}) VisualSfM (\cite{10.1109/3DV.2013.25}), COLMAP)
способны автоматически восстанавливать трёхмерную структуру по произвольному
набору изображений, полученных даже не калиброванными камерами.

\textbf{Преимущества:} методы восстановления структуры из движения позволяют
строить трёхмерные модели по неупорядоченным наборам изображений, не требуя
предварительной калибровки. Они хорошо масштабируются и позволяют обрабатывать
большие массивы фотоданных — вплоть миллиона кадров (например, \cite{10.1109/CVPR.2015.7298949}).

\textbf{Недостатки:} итогом является разреженное представление сцены — т. е.
лишь ограниченное количество пространственных точек. Для получения плотной модели
требуется дальнейшее уплотнение, например, с использованием многовидовой
реконструкции. Кроме того, методы SfM чувствительны к качеству входных
изображений. Если объект однороден, лишён текстурных деталей или является
прозрачным, найти устойчивые соответствия между изображениями становится
затруднительно или невозможно.

\subsection{Фотометрические методы (Photometric Stereo)}
\subsection{Активные методы реконструкции}

\section{Нейросетевые методы 3D-реконструкции}

