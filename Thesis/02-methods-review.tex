\chapter{Теоретический обзор методов восстановления 3D-моделей}

\section{Постановка задачи 3D-восстановления}

Качественно выполненная огранка драгоценного камня существенно повышает
стоимость ювелирного изделия, в то же время неправильная огранка негативно
влияет на эстетическое восприятие. Для минимизации человеческого
фактора и для ускорения процесса желателен нейросетевой метод реконструкции,
который мог бы использовать серию снимков для автоматического получения
высокоточной трехмерной модели камня. Это позволяет мастеру мгновенно сравнивать
получаемый результат с исходной моделью и оперативно корректировать действия для
достижения идеальной формы.

В то же время, в археологии имеются принципиально иные предпосылки для задачи
восстановления трехмерных моделей. Здесь трехмерная реконструкция нужна не
столько для достижения точности соответствия эталонной геометрии, сколько для
идентификации личности по черепу, изучения анатомических особенностей,
проведения антропологических исследований и судебно-медицинских экспертиз. Целью
является реконструкция полной и детализированной модели объекта, даже если
исходные данные неполны или повреждены. Метод должен быть устойчив к шумам и
потерям информации, и при этом позволять визуализировать максимально
информативную трехмерную модель.

Противопоставляя две эти задачи, следует отметить, что восстановление геометрии
драгоценного камня требует высокой точности и низкой погрешности реконструкции. В то же
время, реконструкция археологического объекта, такого как череп, направлена
прежде всего на целостность и реалистичность итоговой модели, которая будет
использоваться в дальнейших экспертных исследованиях и идентификации, где точное
соответствие исходной форме не всегда возможно и не является основной целью.

Таким образом, для эффективного решения обеих задач необходимы различные подходы
к трехмерному моделированию: для огранки камней приоритетом является
высокоточная геометрическая реконструкция, в то время как в археологической
задаче реконструкции черепов важна полная, визуально реалистичная модель,
способная корректно отразить исходную структуру и особенности, даже в условиях
частичной потери исходных данных.

\section{Традиционные и современные подходы}
\section{Нейросетевые методы 3D-реконструкции}

