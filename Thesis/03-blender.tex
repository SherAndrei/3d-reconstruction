\chapter{Генерация синтетических данных для обучения}

\section{Мотивировка использования синтетических данных}

Современные методы обучения с использованием нейросетей требуют значительного
объёма размеченных данных для достижения удовлетворительной точности. Однако
традиционные наборы данных, размеченные вручную, имеют ряд ограничений, особенно
в прикладных инженерных задачах.

\begin{enumerate}
  \item Ручная аннотация даже тысячи изображений требует сотен часов
  человеческого труда, а также подвержена ошибкам, связанным с субъективностью
  разметчика и человеческим фактором.
  \item Стандартные открытые наборы данных, например, для сегментации
  используют COCO\footnote{\url{https://cocodataset.org/}} или
  OpenImages\footnote{\url{https://storage.googleapis.com/openimages/web/index.html}},
  для задач MVS — ShapeNet\footnote{\url{https://shapenet.org/}},
  ABC\footnote{\url{https://deep-geometry.github.io/abc-dataset/}},
  BlendedMVS\footnote{\url{https://github.com/YoYo000/BlendedMVS}},
  DTU\footnote{\url{https://roboimagedata.compute.dtu.dk/}} и другие, не
  содержат узкоспециализированных объектов, необходимых для задач, связанных с
  прозрачными или оптически сложными материалами, такими как стекло или
  драгоценные камни. В большинстве случаев они ориентированы на распознавание
  предметов общего назначения (мебель, бытовая техника и т.п.).
  \item Такие наборы, как правило, не включают точные аннотации глубины,
  параметров камеры или тепловых карт, которые необходимы для задач трёхмерной
  реконструкции и требуют использования дополнительных дорогостоящих сенсоров
  или ручной постобработки.
  \item Наконец, даже при наличии подходящего набора данных, нельзя полагаться
  на постоянный доступ к нему: он может оказаться временно недоступным из-за
  технических сбоев, изменения политики распространения, либо ограничения
  доступа по регионам. Таким образом, отсутствие локальной копии делает
  невозможным воспроизведение эксперимента в любой момент.
\end{enumerate}

Использование синтетических наборов данных позволяет преодолеть
вышеперечисленные ограничения:
\begin{enumerate}
  \item Генерация разметки может быть полностью автоматизирована и выполнена с
  абсолютной точностью: можно получать точные пиксельные маски, карты глубины,
  нормали, координаты ключевых точек, параметры освещения и позы камеры.
  \item Состав сцены и объектов может быть произвольно задан пользователем,
  включая модели, не существующие в реальности, что особенно полезно при
  исследовании теоретических или проектируемых объектов.
  \item В случае ограниченного или нестабильного доступа к интернету
  синтетические данные, созданные локально, обеспечивают независимость от
  внешних хранилищ и повышают воспроизводимость экспериментов.
\end{enumerate}

Таким образом, синтетические данные становятся не просто заменой, а зачастую
необходимым условием для воспроизводимого и точного эксперимента в задачах,
выходящих за рамки типовых сценариев компьютерного зрения.

\section{Выбор инструмента для генерации синтетических данных}

Для получения качественных синтетических изображений с аннотациями глубины,
масок, траекторий камер и прочих параметров требуется инструмент, одновременно
обеспечивающий физически корректный рендеринг, полную управляемость сцены и
возможность масштабируемой автоматизации. Среди всех доступных решений
оптимальным выбором в рамках данной работы является программа
Blender\footnote{\url{https://www.blender.org/}}.

Одним из ключевых преимуществ Blender является его \emph{свободное распространение}:
программа имеет открытую лицензию и полностью бесплатна. Это отличает её от
коммерческих пакетов, таких как Autodesk
Maya\footnote{\url{https://www.autodesk.com/products/maya/overview}}, 3ds
Max\footnote{\url{https://www.autodesk.com/products/3ds-max/overview}} или
рендеринговых движков вроде V-Ray\footnote{\url{https://www.chaos.com/vray}},
лицензии на которые могут стоить значительные суммы.  Открытый исходный код
Blender также означает, что пользователь может детально изучить поведение
системы и при необходимости модифицировать её под собственные задачи.

Вторым важным достоинством является \emph{унификация рабочего процесса}. В одной
программе объединены средства трёхмерного моделирования, система настройки
материалов по физически корректной модели отражения (в частности, поддержка
шейдера Principled
BSDF\footnote{\url{https://docs.blender.org/manual/en/latest/render/shader_nodes/shader/principled.html}}),
средства высокоточного рендеринга с использованием трассировки лучей (движок
Cycles\footnote{\url{https://docs.blender.org/manual/en/latest/render/cycles/index.html}}),
а также возможность быстрого предварительного просмотра сцены (движок
Eevee\footnote{\url{https://docs.blender.org/manual/en/latest/render/eevee/index.html}}).
Это позволяет не переключаться между разными приложениями при подготовке,
проверке и генерации изображений.

Третьим и, пожалуй, наиболее значимым фактором является \emph{автоматизация}.
Все действия, доступные в графическом интерфейсе Blender, могут быть
воспроизведены программно с использованием встроенного языка
Python\footnote{\url{https://www.python.org/}}. Это даёт возможность создавать
сцены, управлять положением камер, освещением и материалами, а также производить
рендеринг и экспорт аннотаций полностью автоматически, в пакетном режиме, в том
числе и на сервере без графического интерфейса (режим ``без головы'', англ.
headless). Такая гибкость делает Blender исключительно подходящим для генерации
больших объёмов данных с заданной структурой и повторяемостью.

Наконец, Blender — это кроссплатформенное решение, доступное для операционных
систем Windows, Linux и macOS. Проект поддерживается сообществом и развиваем при
участии ведущих технологических компаний, что обеспечивает устойчивое развитие и
быстрое внедрение современных графических технологий.

Альтернативные решения, доступные на рынке, в сравнении с Blender имеют
существенные ограничения. Так, игровые движки, такие как
Unity\footnote{\url{https://unity.com/}} или Unreal
Engine\footnote{\url{https://www.unrealengine.com/en-US}}, избыточны по
возможностям при решении задач генерации статичных изображений. Они часто
сопровождаются сложными условиями лицензирования и могут требовать платных
подписок или предварительного коммерческого согласования.  Более того, их
архитектура ориентирована в первую очередь на интерактивные сцены, а не на
контроль точности визуализации.

Есть и специализированные инструменты, такие как
UnrealCV\footnote{\url{https://unrealcv.org/}},
SynthEyes\footnote{\url{https://borisfx.com/products/syntheyes/}} или другие
пакеты для визуального трекинга и аннотирования. Однако большинство из них либо
ограничены по функциональности (например, работают только с одной моделью
камеры), либо сложны в интеграции и настройке.

Таким образом, Blender в силу своей универсальности, автоматизируемости,
доступности и поддержки профессионального рендеринга представляет собой наиболее
рациональный выбор для создания синтетических наборов данных в рамках задачи
реконструкции построения сложных геометрических фигур.

