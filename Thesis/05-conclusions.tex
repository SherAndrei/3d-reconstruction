\chapter{Заключение}

В данной работе была поставлена цель разработать методику восстановления
трёхмерных моделей объектов в сложных сценах. Для её
достижения были сформулированы три задачи: аналитический обзор методов
3D-реконструкции, поиск и подготовка подходящих наборов данных, а также
экспериментальное сравнение традиционных и нейросетевых алгоритмов на этих
данных. В рамках исследования все перечисленные задачи выполнены.

Во‑первых, проведён подробный обзор отечественных и зарубежных публикаций,
охватывающий эволюцию методов от стереозрения и MVS до генеративных нейросетевых
моделей и неявных представлений поверхностей. Анализ позволил систематизировать
преимущества и ограничения подходов и выделить критерии, критически важные для
сложных сцен: требуемый объём входных данных, точность геометрии, устойчивость
к шуму и время вычислений.

Во‑вторых, решена проблема дефицита открытых датасетов для сцен с богатой
геометрией и неконтролируемыми условиями съёмки. Для этого разработан
конфигурируемый скрипт \texttt{generate-batch}, использующий Blender, с
системой плагинов, автоматически генерирующий изображения, глубину, нормали,
маски и полные параметры камеры. Дополнительный скрипт \texttt{to\_colmap}
переводит синтетические в формат COLMAP, что обеспечивает совместимость с
классическими алгоритмами SfM. Открытость и документированность инструментов
позволяют быстро формировать собственные выборки и повторять эксперименты.

В‑третьих, проведена серия воспроизводимых экспериментов. Классический конвейер
SfM+MVS и современная модель TensoRF протестированы на двух типах данных: (i)
синтетических сценах с огранённым драгоценным камнем и археологическим
объектом; (ii) наборе фотографий, снятых смартфоном. Полученные результаты
показывают, что:

\begin{itemize}
  \item при достаточном числе ракурсов и точной калибровке камеры TensoRF даёт заметно
  более высокие показатели PSNR по сравнению с MVS, позволяя восстановить
  мелкие детали и сократить ручную пост‑обработку;
  \item в условиях ограниченного набора снимков и отсутствия масок классическая
  фотограмметрия остаётся конкурентоспособной по времени и стойкости
  к артефактам;
  \item использование синтетических фотосессий, сгенерированных описанным скриптом,
  упрощает тонкую настройку нейросетевой модели и позволяет уменьшить объём
  реальных данных, необходимых для обучения.
\end{itemize}

Таким образом, предложена практическая методика, подтверждённая экспериментами,
и создан набор инструментов для её дальнейшего применения и развития.

Инструментарий и рекомендации, разработанные в работе, могут быть использованы
для оценки качества огранки драгоценных камней, цифровой архивации
археологических находок, а также в смежных областях, где требуется точное или
быстродействующее восстановление геометрии при ограниченном доступе
к сканирующему оборудованию.

Полученная методика пока не рассматривает прозрачные и сильно отражающие
поверхности, а также не оптимизирована под реконструкцию в реальном времени.

Реализация этих направлений позволит ещё более широко применять предложенную
методику в практике инженерных, культурно‑исторических и медицинских задач.
