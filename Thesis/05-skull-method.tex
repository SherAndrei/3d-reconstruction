\chapter{Метод для точного восстановления черепов}

\section{Выбор модели}

Для археологических задач реконструкции черепов нейросетевая архитектура MVSNet
представляет собой оптимальный выбор благодаря сочетанию нескольких ключевых
характеристик. Архитектура изначально разрабатывалась и тестировалась на наборах
из 5-7 изображений, что идеально соответствует реалиям полевой археологии, где
редко удаётся сделать более 10-15 качественных снимков объекта.

Важным практическим преимуществом является наличие открытых предобученных
моделей, доступных для наборов данных DTU и Tanks \& Temples. Это позволяет
исследователям сразу применять метод без трудоёмкого этапа сбора обучающей
выборки и тренировки модели с нуля, что особенно ценно при работе с уникальными
археологическими находками, где создание референсных 3D-моделей может быть
затруднено.

\section{Обзор модели}
\section{Подбор гиперпараметров}
\section{Обучение и запуск модели на собственных данных}

