\chapter{Введение}

Восстановление трехмерных моделей (далее — 3D-моделей) — одна из наиболее
популярных задач компьютерного зрения и графики. Цифровое представление и
воссоздание трехмерных сцен на компьютере лежит в основе множества важных
приложений, а в последние годы 3D-реконструкция стала особенно актуальной
благодаря увеличению вычислительных мощностей графических ускорителей и
последующему ускоренному развитию машинного обучения и нейронных сетей.
Современные методы способны получать объемные модели высокой точности из
ограниченного набора изображений (далее — датасетов), что открывает пути к
упрощенному созданию фотореалистичных виртуальных миров и цифровых копий
реальных объектов.

Задача автоматического построения 3D-модели по ограниченному количеству снимков
крайне востребована: она позволяет отказаться от ручного моделирования и
непосредственного физического копирования объектов. Технологии 3D-реконструкции
находят применение в самых разных сферах: от развлечений (игры, кино,
виртуальная реальность), медицины (моделирование органов по снимкам для
подготовки операций), робототехники (построение карты окружения для навигации)
до промышленного дизайна и инженерии (создание точных цифровых моделей изделий
для проектирования). Таким образом, тема реконструкции 3D-моделей с помощью
нейросетей является актуальной благодаря как научному интересу, так и огромному
практическому потенциалу во множестве отраслей.

Целью данной работы является построение методики восстановления трёхмерных
моделей объектов в сложных сценах, основанной на сочетании классических
фотограмметрических алгоритмов и современных нейросетевых подходов.

Задачи включают в себя:
\begin{enumerate}
  \item обзор отечественных и зарубежных публикаций, описывающих
  3D-реконструкцию — от традиционных (стереозрение, многовидовая реконструкция,
  фотометрические и активные методы сканирования) до нейросетевых моделей
  последнего поколения (генеративные модели, обучаемые многовидовые методы,
  нейронные радиационные поля), оценка преимуществ и ограничений
  различных подходов.
  \item поиск открытых наборов данных, пригодных для обучения и тестирования
  алгоритмов на сценах со сложной геометрией и неидеальными условиями съёмки.
  \item проведение серии экспериментов с классическими и нейросетевыми методами
  на найденных наборах данных и на собственно снятых телефонных фотографиях,
  оценка получаемого качества реконструкции.
\end{enumerate}

Таким образом, исследование позволит получить представление о возможностях
современных нейросетевых подходов к 3D-реконструкции, определить оптимальные
алгоритмы для решения практических задач, а также выявить направления для
дальнейших улучшений и развития методов автоматического восстановления
трехмерных сцен.
