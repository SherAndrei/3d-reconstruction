\chapter{Введение}

% базовое введение, определение задачи
Восстановление трехмерных моделей (далее — 3D-моделей) — одна из наиболее
популярных задач компьютерного зрения и графики. Цифровое представление и
воссоздание трехмерных сцен на компьютере лежит в основе множества важных
приложений, а в последние годы 3D-реконструкция стала особенно актуальной
благодаря увеличению вычислительных мощностей графических ускорителей и
последующему ускоренному развитию машинного обучения и нейронных сетей.
Современные методы способны получать объемные модели высокой точности из
ограниченного набора изображений (далее — датасетов), что открывает пути к
упрощенному созданию фотореалистичных виртуальных миров и цифровых копий
реальных объектов.

% востребованность восстановления 3д моделей по всему миру
Задача автоматического построения 3D-модели по ограниченному количеству снимков
крайне востребована: она позволяет отказаться от ручного моделирования и
непосредственного копирования объектов.  В ряде случаев методы трехмерного
восстановления выступают альтернативой физическому копированию ценных и хрупких
артефактов культурного наследия. Технологии 3D-воссоздания находят применение в
самых разных сферах: от развлечений (игры, кино, виртуальная реальность),
медицины (моделирование органов по снимкам для подготовки операций),
робототехники (построение карты окружения для навигации) до промышленного
дизайна и инженерии (создание точных цифровых моделей изделий для
проектирования). Таким образом, тема реконструкции 3D-моделей с помощью
нейросетей является актуальной благодаря как научному интересу, так и огромному
практическому потенциалу во множестве отраслей.

% NeRF кратко
Прорывным достижением в области является метод NeRF
(\cite{mildenhall2020nerf}). Его авторы достигли успехов в задаче
реконструкции сложных сцен, задавая сцену непрерывной функцией,
аппроксимируемой нейросетью. % TODO: сослаться ниже
Тем не менее, остаются открытыми вопросы оптимизации скорости и обобщаемости таких моделей.

В свете быстрых темпов исследований и появления разнообразных подходов назрела
необходимость сравнительного анализа: какие методы восстановления 3D-моделей с
помощью нейросетей существуют, в чем их особенности, и как выбрать оптимальный
подход для решения практической задачи. Это определяет актуальность проведенного
исследования и разработки, направленной на реализацию и сопоставление
возможностей NeRF и альтернативного метода.

Целью работы является исследование современных нейросетевых методов 3D-реконструкции
и разработка на их основе подхода к восстановлению трехмерных моделей,
обеспечивающего высокое качество результирующих сцен при разумных затратах
вычислительных ресурсов. Достижение поставленной цели предусматривает решение
ряда взаимосвязанных задач:

\begin{enumerate}

	\item \textbf{Анализ литературы и существующих методов.} Изучить текущее состояние
	области 3D-восстановления с помощью нейронных сетей, включая как обзорные
	источники, так и конкретные передовые разработки. Систематизировать подходы,
	их преимущества и ограничения.
	\item \textbf{Выбор методов для реализации.} Выбрать два контрастирующих
	нейросетевых метода 3D-реконструкции для дальнейшей практической реализации и
	сравнения в рамках работы. Обосновать выбор каждого из методов.
	\item \textbf{Реализация алгоритмов.} Реализовать (или адаптировать
	существующие реализации) выбранные методы на тестовых данных: воспроизвести
	алгоритмы для заданного набора изображений.
	\item \textbf{Проведение
	экспериментов.} Выполнить экспериментальное сравнение методов на общих данных:
	оценить качество восстановленной геометрии и фотореалистичность визуализации,
	требования к объему исходных данных, время обучения модели и генерации
	результатов.
	\item \textbf{Анализ результатов.} Проанализировать полученные
	результаты, определить сильные и слабые стороны каждого из методов. Сделать
	выводы о том, какой подход в наибольшей степени отвечает поставленной цели и
	как их можно комбинировать или улучшить.
\end{enumerate}

Таким образом, исследование позволит получить представление о
возможностях современных нейросетевых подходов к 3D-реконструкции, определить
оптимальные алгоритмы для решения практических задач, а также выявить
направления для дальнейших улучшений и развития методов автоматического
восстановления трехмерных сцен.
