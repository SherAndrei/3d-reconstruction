\chapter{Введение}

Восстановление трехмерных моделей (далее — 3D-моделей) — одна из наиболее
популярных задач компьютерного зрения и графики. Цифровое представление и
воссоздание трехмерных сцен на компьютере лежит в основе множества важных
приложений, а в последние годы 3D-реконструкция стала особенно актуальной
благодаря увеличению вычислительных мощностей графических ускорителей и
последующему ускоренному развитию машинного обучения и нейронных сетей.
Современные методы способны получать объемные модели высокой точности из
ограниченного набора изображений (далее — датасетов), что открывает пути к
упрощенному созданию фотореалистичных виртуальных миров и цифровых копий
реальных объектов.

Задача автоматического построения 3D-модели по ограниченному количеству снимков
крайне востребована: она позволяет отказаться от ручного моделирования и
непосредственного физического копирования объектов. Технологии 3D-реконструкции
находят применение в самых разных сферах: от развлечений (игры, кино,
виртуальная реальность), медицины (моделирование органов по снимкам для
подготовки операций), робототехники (построение карты окружения для навигации)
до промышленного дизайна и инженерии (создание точных цифровых моделей изделий
для проектирования). Таким образом, тема реконструкции 3D-моделей с помощью
нейросетей является актуальной благодаря как научному интересу, так и огромному
практическому потенциалу во множестве отраслей.

В настоящее время существуют различные подходы к 3D-реконструкции: традиционные
методы (стереозрение, многовидовая реконструкция, фотометрические и активные
методы сканирования) и современные нейросетевые методы (генеративные модели,
обучаемые многовидовые методы, нейронные радиационные поля). Каждый из этих
подходов обладает своими преимуществами и ограничениями, которые важно учитывать
при выборе оптимального решения.

Целью данной работы является исследование современных нейросетевых методов
3D-реконструкции и разработка эффективного подхода для решения задач
с различными требованиями к точности и доступности вспомогательного
оборудования. Достижение поставленной цели предусматривает решение ряда
взаимосвязанных задач:

\begin{enumerate}
	\item Проанализировать текущее состояние области 3D-восстановления, описать
	традиционные методы и методы, использующие нейронные сетей.  Рассмотреть
	обзорные источники, так и конкретные передовые разработки. Систематизировать
	подходы, их преимущества и недостатки.
	\item Выбрать два нейросетевых метода 3D-реконструкции для двух
	контрастирующих задач. Обосновать выбор каждого из методов.
	\item Подготовиться к использованию выбранных методов. Если потребуется,
	сгенерировать соответствующие данных для обучения и подобрать гиперпараметры
	нейронной сети.
	\item Провести эксперимент, оценить качество восстановленной геометрии и
	фотореалистичность визуализации, требования к объему исходных данных, время
	обучения модели и генерации результатов.
	\item Проанализировать полученные
	результаты, определить сильные и слабые стороны каждого из методов. Сделать
	выводы о том, какой подход в наибольшей степени отвечает поставленной цели и
	как их можно комбинировать или улучшить.
\end{enumerate}

Таким образом, исследование позволит получить представление о
возможностях современных нейросетевых подходов к 3D-реконструкции, определить
оптимальные алгоритмы для решения практических задач, а также выявить
направления для дальнейших улучшений и развития методов автоматического
восстановления трехмерных сцен.
